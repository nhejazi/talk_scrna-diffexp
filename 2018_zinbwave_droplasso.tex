\documentclass{beamer}
%
% Choose how your presentation looks.
%
% For more themes, color themes and font themes, see:
% http://deic.uab.es/~iblanes/beamer_gallery/index_by_theme.html
%
\mode<presentation>
{
  \usetheme{default}      % or try Darmstadt, Madrid, Warsaw, ...
  \usecolortheme{default} % or try albatross, beaver, crane, ...
  \usefonttheme{default}  % or try serif, structurebold, ...
  \setbeamertemplate{navigation symbols}{}
  \setbeamertemplate{caption}[numbered]
}

\usepackage[english]{babel}
\usepackage[utf8x]{inputenc}

\usepackage{amstext}
%\usepackage{coloremoji}

\usepackage{graphicx}
\graphicspath{ {figs/} }

\setbeameroption{hide notes}
\setbeamertemplate{note page}[plain]
\usepackage{listings}
\usepackage{datetime}
\usepackage{url}

% specifications for presenter mode
%\beamerdefaultoverlayspecification{<+->}
%\setbeamercovered{transparent}

% math shorthand
\usepackage{bm}
\usepackage{amsmath}
\usepackage{mathtools}
\newcommand{\D}{\mathcal{D}}
\newcommand{\E}{\mathbb{E}}
\newcommand{\F}{\mathcal{F}}
\newcommand{\X}{\mathcal{X}}
\newcommand{\lik}{\mathcal{L}}
\DeclarePairedDelimiterX{\infdivx}[2]{(}{)}{%
  #1\;\delimsize\|\;#2%
}
\newcommand{\infdiv}{D\infdivx}
\DeclarePairedDelimiter{\norm}{\lVert}{\rVert}
\DeclareMathOperator*{\argmin}{arg\,min}
\DeclareMathOperator*{\argmax}{arg\,max}

% Bibliography
\usepackage{natbib}
\bibpunct{(}{)}{,}{a}{}{;}
\usepackage{bibentry}

%\nobibliography*
\title[zinbwave-droplasso]{Differential Expression Analysis for Single-Cell
  RNA-seq Data}
\subtitle{\vspace*{0.5em} \scriptsize for the \textit{Computational Biology
  Doctoral Seminar},\\ organized by N.~Yosef \& T.~Ashuach, Spring 2018,
  UC Berkeley}
\author{Kevin Benac and Nima Hejazi}
\institute{Group in Biostatistics,\\ University of California, Berkeley}
\date{05 April 2018}

%%%%%%%%%%%%%%%%%%%%%%%%%%%%%%%%%%%%%%%%%%%%%%%%%%%%%%%%%%%%%%%%%%%%%%%%%%%%%%%%

\begin{document}

\begin{frame}
  \titlepage
\end{frame}

%%%%%%%%%%%%%%%%%%%%%%%%%%%%%%%%%%%%%%%%%%%%%%%%%%%%%%%%%%%%%%%%%%%%%%%%%%%%%%%%

\begin{frame}{Preview}

\begin{itemize}
  \itemsep10pt
  \item ...
  \item ...
  \item ...
\end{itemize}

\end{frame}

%%%%%%%%%%%%%%%%%%%%%%%%%%%%%%%%%%%%%%%%%%%%%%%%%%%%%%%%%%%%%%%%%%%%%%%%%%%%%%%%

\begin{frame}{The Data and Objective}

\begin{itemize}
  \itemsep10pt
  \item ...
\end{itemize}

\end{frame}

%%%%%%%%%%%%%%%%%%%%%%%%%%%%%%%%%%%%%%%%%%%%%%%%%%%%%%%%%%%%%%%%%%%%%%%%%%%%%%%%

\begin{frame}{Review}

\begin{itemize}
  \itemsep10pt
  \item ...
  \item ...
\end{itemize}

\end{frame}

%%%%%%%%%%%%%%%%%%%%%%%%%%%%%%%%%%%%%%%%%%%%%%%%%%%%%%%%%%%%%%%%%%%%%%%%%%%%%%%%

% don't want dimming with references
\setbeamercovered{}
\beamerdefaultoverlayspecification{}

\begin{frame}[c,allowframebreaks]{References}

\bibliographystyle{apalike}
\nocite{*}
\bibliography{refs}

\end{frame}

%%%%%%%%%%%%%%%%%%%%%%%%%%%%%%%%%%%%%%%%%%%%%%%%%%%%%%%%%%%%%%%%%%%%%%%%%%%%%%%%

\end{document}

